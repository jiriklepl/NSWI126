\documentclass[aspectratio=169]{beamer}
% (the default is 43, i.e. 4:3)


\usepackage[T1]{fontenc}
\usepackage[utf8]{inputenc}
\usepackage{lmodern}

\usepackage{tikz}\usetikzlibrary{positioning}
\usepackage{xcolor}

\usepackage{listings}

\usetheme{metropolis}


\usepackage{FiraSans}

% colored underline, with Beamer overlay support
% usage: \cul{x} or \cul[blue]{x} or \cul<2->{x} or \cul<2->[blue]{x}
\newcommand<>{\cul}[2][red]{%
  % change underline dimentions: https://tex.stackexchange.com/a/167957/25264
  \fontdimen8\textfont3=0.75pt%
  % colored underline: https://tex.stackexchange.com/a/9477/25264
  % transparent underline: https://tex.stackexchange.com/a/45601/25264
  % switch between colored and transparent: http://mirrors.ibiblio.org/CTAN/macros/latex/contrib/beamer/doc/beameruserguide.pdf sections 9.3 and 9.6.1
  \alt#3%
      {\color{#1}\underline{{\color{black}#2}}\color{black}}%
      {\transparent{0.0}\underline{{\transparent{1.0}#2}}\transparent{1.0}}%
}

\title{APT (Advanced Package Tool)}
\author{Jiří Klepl}
\date{}
\begin{document}
\maketitle

\section{Package managers}

\begin{frame}{Proč chtít package manager}
\begin{itemize}
\item \textbf{Z pohledu uživatele}
  \begin{itemize}
    \item Bezpečnost
    \item Snadnost instalace \& updatů
    \item Jednoduché hledání
    \item Přehled o nainstalovaném softwaru
    \item Udržení kompatibility mezi různými programy
  \end{itemize}
\item \textbf{Z pohledu developera}
\begin{itemize}
  \item Přívětivost pro uživatele
  \item Vyšší důvěryhodnost (má to úroveň)
  \item Možnost vytvoření ekologie % TODO
\end{itemize}
\end{itemize}
\end{frame}

\begin{frame}{Package managery}
\begin{itemize}
\item \textbf{apt}: Debian/Ubuntu
\item \textbf{dnf}: Red Hat/Fedora
\item \textbf{zypper}: SLES/openSUSE
\item \textbf{pacman}: Arch
\item \textbf{emerge}: Gentoo
\end{itemize}

Podrobné srovnání: https://wiki.archlinux.org/index.php/Pacman/Rosetta
\end{frame}

\begin{frame}[fragile]{APT - instalační příkazy}
  \begin{itemize}
    \item Více než 50'000 packagů (Debian), 60'000 (Ubuntu)
    \item Instalace
    \begin{lstlisting}
      sudo apt install <package>
    \end{lstlisting}
    \item Odstranění
    \begin{lstlisting}
      sudo apt remove <package>
    \end{lstlisting}
    \item Update \& upgrade
    \begin{lstlisting}
      sudo apt update && sudo apt upgrade
    \end{lstlisting}
  \end{itemize}
\end{frame}

\begin{frame}[fragile]{APT - příkazy pro dotazování}
  \begin{itemize}
    \item Hledání package
    \begin{lstlisting}
      apt search "regex"
    \end{lstlisting}
    \item Informace o packagi
    \begin{lstlisting}
      apt info <package>
    \end{lstlisting}
    \item Seznam nainstalovaných packagů
    \begin{lstlisting}
      apt list --installed
    \end{lstlisting}
    \item Seznam ručně nainstalovaných packagů
    \begin{lstlisting}
      apt list --manual-installed
    \end{lstlisting}
  \end{itemize}
\end{frame}

\begin{frame}[fragile]{Přidání dalších packagů}
  Jednoduše:
  \begin{lstlisting}
    sudo add-apt-repository "deb <repository>"
    sudo apt-key adv --fetch-keys <url>
    sudo apt-get update
  \end{lstlisting}
  Nebo:
  \begin{lstlisting}
    sudo <oblibeny editor (napr. VIM)> /etc/apt/sources.list
    curl -L <url> | sudo apt-key add -
  \end{lstlisting}
\end{frame}

\begin{frame}[fragile]{Zajímavé repozitáře}
  \begin{itemize}
    \item cuda toolkit

    https://docs.nvidia.com/cuda/cuda-installation-guide-linux/index.html

    \item intel oneAPI

    https://software.intel.com/content/www/us/en/develop/articles/installing-intel-oneapi-toolkits-via-apt.html
  \end{itemize}
\end{frame}

\begin{frame}[fragile]{Kde brát informace}
  \begin{itemize}
    \item https://www.debian.org/doc/manuals/packaging-tutorial/packaging-tutorial.en.pdf
    \item https://www.debian.org/doc/debian-policy/index.html
  \end{itemize}
\end{frame}

\begin{frame}[plain]
\centering
{\Large\bfseries Thank you for attention!}
\end{frame}
\end{document}
