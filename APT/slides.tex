\documentclass[aspectratio=169]{beamer}
% (the default is 43, i.e. 4:3)


\usepackage[T1]{fontenc}
\usepackage[utf8]{inputenc}
\usepackage{lmodern}

\usepackage{tikz}\usetikzlibrary{positioning}
\usepackage{xcolor}

\usepackage{listings}

\usetheme{metropolis}


\usepackage{FiraSans}

% colored underline, with Beamer overlay support
% usage: \cul{x} or \cul[blue]{x} or \cul<2->{x} or \cul<2->[blue]{x}
\newcommand<>{\cul}[2][red]{%
  % change underline dimentions: https://tex.stackexchange.com/a/167957/25264
  \fontdimen8\textfont3=0.75pt%
  % colored underline: https://tex.stackexchange.com/a/9477/25264
  % transparent underline: https://tex.stackexchange.com/a/45601/25264
  % switch between colored and transparent: http://mirrors.ibiblio.org/CTAN/macros/latex/contrib/beamer/doc/beameruserguide.pdf sections 9.3 and 9.6.1
  \alt#3%
      {\color{#1}\underline{{\color{black}#2}}\color{black}}%
      {\transparent{0.0}\underline{{\transparent{1.0}#2}}\transparent{1.0}}%
}

\title{APT (Advanced Package Tool)}
\author{Jiří Klepl}
\date{}
\begin{document}
\maketitle

\begin{frame}{Proč chtít package manager}
  \begin{itemize}
    \item \textbf{Z pohledu uživatele}
    \begin{itemize}
      \item Snadnost instalace \& updatů
      \item Jednoduché hledání
      \item Přehled o nainstalovaném softwaru
      \item Udržení kompatibility mezi různými programy
      \item Bezpečnost
    \end{itemize}
    \item \textbf{Z pohledu developera}
    \begin{itemize}
      \item Přívětivost pro uživatele
      \item Vyšší důvěryhodnost (má to úroveň)
      \item Možnost vytvoření ekosystému
    \end{itemize}
  \end{itemize}
\end{frame}

\begin{frame}{Package managery}
  \begin{itemize}
    \item \textbf{apt}: Debian/Ubuntu
    \item \textbf{dnf}: Red Hat/Fedora
    \item \textbf{zypper}: SLES/openSUSE
    \item \textbf{pacman}: Arch
    \item \textbf{emerge}: Gentoo
  \end{itemize}

Podrobné srovnání: https://wiki.archlinux.org/index.php/Pacman/Rosetta
\end{frame}

\begin{frame}[fragile]{APT - Advanced Package Tool}
  \begin{itemize}
    \item Package manager pro debian (a ubuntu)
    \item Skupina více programů: \textbf{apt}, \textbf{apt-get}, \textbf{apt-cache}
    \item Frontend pro \textbf{dpkg}: low-lvl nástroj pro instalaci a správu packagů
  \end{itemize}
\end{frame}

\begin{frame}[fragile]{APT - instalační příkazy}
  \begin{itemize}
    \item Více než 50'000 packagů (Debian), 60'000 (Ubuntu)
    \item Instalace
    \begin{lstlisting}
  sudo apt install <package>
    \end{lstlisting}
    \item Odstranění
    \begin{lstlisting}
  sudo apt remove <package>
    \end{lstlisting}
    \item Update \& upgrade
    \begin{lstlisting}
  sudo apt update && sudo apt upgrade
    \end{lstlisting}
  \end{itemize}
\end{frame}

\begin{frame}[fragile]{APT - příkazy pro dotazování}
  \begin{itemize}
    \item Hledání package
    \begin{lstlisting}
  apt search "regex"
    \end{lstlisting}
    \item Informace o packagi
    \begin{lstlisting}
  apt info <package>
    \end{lstlisting}
    \item Seznam nainstalovaných packagů
    \begin{lstlisting}
  apt list --installed
    \end{lstlisting}
    \item Seznam ručně nainstalovaných packagů
    \begin{lstlisting}
  apt list --manual-installed
    \end{lstlisting}
  \end{itemize}
\end{frame}

\begin{frame}[plain]
  \centering
  {\Large\bfseries Příklad: ./examples.sh}
\end{frame}

\begin{frame}[fragile]{Přidání dalších packagů}
  Jednoduše:
  \begin{lstlisting}
  sudo add-apt-repository "deb <repository>"
  sudo apt-key adv --fetch-keys <url>
  sudo apt-get update
  \end{lstlisting}
  Nebo:
  \begin{lstlisting}
  sudo <oblibeny editor (napr. VIM)> /etc/apt/sources.list
  curl -L <url> | sudo apt-key add -
  \end{lstlisting}
\end{frame}

\begin{frame}[fragile]{Zajímavé repozitáře}
  \begin{itemize}
    \item cuda toolkit

    https://docs.nvidia.com/cuda/cuda-installation-guide-linux/index.html

    \item intel oneAPI

    https://software.intel.com/content/www/us/en/develop/articles/installing-intel-oneapi-toolkits-via-apt.html
  \end{itemize}
\end{frame}

\begin{frame}[fragile]{Co to je package}
  Dva druhy
  \begin{itemize}
    \item \textbf{binary package}
    \begin{itemize}
      \item záznam repozitáře začíná ``deb'',
      
      např. deb https://deb.debian.org/debian bullseye main
      \item pro koncové uživatele
      \item obecně jen pro určité architektury
    \end{itemize}
    \item \textbf{source package}
    \begin{itemize}
      \item záznam repozitáře začíná ``deb-src'',
      
      např. deb-src https://deb.debian.org/debian bullseye main
      \item pro development, testing a customizaci
    \end{itemize}
  \end{itemize}
\end{frame}

\begin{frame}[fragile]{Z čeho se skládá package připraven pro sestavení}
  \begin{itemize}
    \item složka \textbf{debian} ve složce \textbf{<název>-<verze (major.minor.patch)>} se zdrojem programu; obsah \textbf{debian}:
    \begin{itemize}
      \item \textbf{changelog}: informace o verzích package
      \item \textbf{rules}: script pro sestavení package (např. makefile - může pak spustit tradiční makefile programu; tato část je extrémě flexibilní - zde použijeme makefile s dh\_helpery, co řeší boilerplate)
      \item \textbf{control}: různé informace o package
      \begin{itemize}
        \item název, aktuální verze a kategorie (sekce a priorita (důležitost pro systém))
        \item maintainer a homepage
        \item popis programu
        \item architektura (např. amd64, ale i all nebo any)
        \item informace o (build-) dependencích a vyloučení
      \end{itemize}
      \item \textbf{compat}: verze dh\_helperů
    \end{itemize}
  \end{itemize}
\end{frame}

\begin{frame}[plain]
  \centering
  {\Large\bfseries Příklad: ../packages/myhello-1.0.0/debian}
\end{frame}

\begin{frame}[fragile]{Sestavení package}
  \begin{itemize}
    \item \lstinline{debuild -b} sestaví podepsaný .deb package
    \item \lstinline{debuild -b -us -uc} sestaví nepodepsaný .deb package pro vlastní použití
    \item složna \textbf{debian} simuluje root file-systému
      \begin{itemize}
        \item binárka programu tradičně v \textbf{/usr/bin/}, tedy při sestavování package: \textbf{debian/usr/bin/}
        \item library programu tradičně v \textbf{/usr/lib.../}, tedy při sestavování package: \textbf{debian/usr/lib.../}
      \end{itemize}
    \item vznikne .deb archív a doprovodné soubory
    \item název vzniklého archívu obsahuje verzi, číslo revize a architekturu
  \end{itemize}
\end{frame}

\begin{frame}[plain]
  \centering
  {\Large\bfseries Příklad: ../packages/myhello-1.0.0}
\end{frame}

\begin{frame}[fragile]{.deb soubor: \texttt{ar} archive}
  \begin{lstlisting}
$ ar tv wget_1.12-2.1_i386.deb
rw-r--r-- 0/0 4 Sep 5 15:43 2010 debian-binary
rw-r--r-- 0/0 2403 Sep 5 15:43 2010 control.tar.gz
rw-r--r-- 0/0 751613 Sep 5 15:43 2010 data.tar.gz
  \end{lstlisting}
  \begin{itemize}
    \item \texttt{debian-binary}: version of the deb file format, "2.0{\textbackslash}n"
    \item \texttt{control.tar.gz}: metadata about the package
    \item \texttt{control, md5sums, (pre|post)(rm|inst), triggers, shlibs},...
    \item \texttt{data.tar.gz}: data files of the package
  \end{itemize}
  from: https://www.debian.org/doc/manuals/packaging-tutorial/packaging-tutorial.en.pdf
\end{frame}

\begin{frame}[fragile]{Co když mám důvěryhodný program, který není .deb package}
  Lze jednoduše zabalit do .deb package a nainstalovat přes dpkg / apt (opět příklad myhello)

  Proč to dělat?
  \begin{itemize}
    \item Konzistence se zbytkem systému (jednotný způsob instalace)
    \item Chci mít přehled i o softwaru, který není z repozitářů pro můj OS
    \item Vyhnutí se mrtvým souborům v systému
    \item Vyhnutí se mrtvým souborům v systému
  \end{itemize}
\end{frame}

\begin{frame}{Apt repozitář}
  Složka s .deb packagy a pomocnými soubory:
  \begin{itemize}
    \item \texttt{Packages}: soubor se seznamem packagů, stručnými informacemi a kontrolními součty
    \item \texttt{Packages.gz} (optional): komprimovaný soubor Packages
    \item \texttt{Release}: soubor s kontrolními součty pro soubory Packages (,Packages.gz) a sebe
    \item \texttt{Release.gpg}: gnupg podpis souboru Release
    \item \texttt{InRelease}: Release a Release.gpg dohromady
    \item \texttt{public.key} (pokud nelze předat jiným způsobem): veřejný klíč pro ověření
  \end{itemize}
\end{frame}

\begin{frame}[plain]
  \centering
  {\Large\bfseries Příklad:}

  {\Large\bfseries ../packages/scripts.sh}

  {\Large\bfseries "deb https://klepl.cz/packages /"}
\end{frame}

\begin{frame}[fragile]{Kde vzít další informace}
  \begin{itemize}
    \item https://www.debian.org/doc/manuals/packaging-tutorial/packaging-tutorial.en.pdf
    \item https://www.debian.org/doc/debian-policy/index.html
    \item https://medium.com/sqooba/create-your-own-custom-and-authenticated-apt-repository-1e4a4cf0b864
    \item https://wiki.archlinux.org/index.php/Pacman/Rosetta
  \end{itemize}
\end{frame}

\begin{frame}[plain]
  \centering
  {\Large\bfseries Univerzum pro dotazy}
\end{frame}

\begin{frame}[plain]
  \centering
  {\Large\bfseries Díky za pozornost!}
\end{frame}

\end{document}
  